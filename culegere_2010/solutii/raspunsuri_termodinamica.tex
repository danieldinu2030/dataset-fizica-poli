%% Raspunsuri Fizica Moleculara si Termodinamica %%

\section*{2. FIZICĂ MOLECULARĂ ŞI TERMODINAMICĂ}
\begin{center}
\begin{tabular}{|l|l|l|l|l|l|}
\hline
2.1 - B & 2.22 - A & 2.43 - D & 2.64 - A & 2.85 - A & 2.106 - F \\
\hline
2.2 - A & 2.23 - B & 2.44 - D & 2.65 - C & 2.86 - A & 2.107 - B \\
\hline
2.3 - E & 2.24 - D & 2.45 - D & 2.66 - B & 2.87 - A & 2.108 - E \\
\hline
2.4 - C & 2.25 - C & 2.46 - C & 2.67 - C & 2.88 - A & 2.109 - A \\
\hline
2.5 - D & 2.26 - E & 2.47 - F & 2.68 - F & 2.89 - A & 2.110 - D \\
\hline
2.6 - F & 2.27 - F & 2.48 - A & 2.69 - A & 2.90 - C & 2.111 - C \\
\hline
2.7 - B & 2.28 - B & 2.49 - B & 2.70 - F & 2.91 - A & 2.112 - B \\
\hline
2.8 - D & 2.29 - C & 2.50 - B & 2.71 - D & 2.92 - A & 2.113 - E \\
\hline
2.9 - A & 2.30 - D & 2.51 - E & 2.72 - F & 2.93 - C & 2.114 - D \\
\hline
2.10 - C & 2.31 - F & 2.52 - C & 2.73 - A & 2.94 - A & 2.115 - C \\
\hline
2.11 - F & 2.32 - C & 2.53 - F & 2.74 - F & 2.95 - A & 2.116 - D \\
\hline
2.12 - E & 2.33 - E & 2.54 - B & 2.75 - A & 2.96 - A & 2.117 - C \\
\hline
2.13 - A & 2.34 - A & 2.55 - A & 2.76 - E & 2.97 - B & 2.118 - C \\
\hline
2.14 - F & 2.35 - D & 2.56 - C & 2.77 - C & 2.98 - A & 2.119 - B \\
\hline
2.15 - B & 2.36 - C & 2.57 - B & 2.78 - A & 2.99 - B & 2.120 - B \\
\hline
2.16 - B & 2.37 - B & 2.58 - C & 2.79 - B & 2.100 - E & 2.121 - E \\
\hline
2.17 - E & 2.38 - E & 2.59 - B & 2.80 - B & 2.101 - D & 2.122 - C \\
\hline
2.18 - D & 2.39 - B & 2.60 - D & 2.81 - C & 2.102 - A & 2.123 - D \\
\hline
2.19 - B & 2.40 - D & 2.61 - C & 2.82 - E & 2.103 - F & 2.124 - B \\
\hline
2.20 - A & 2.41 - A & 2.62 - E & 2.83 - E & 2.104 - D & 2.125 - D \\
\hline
2.21 - C & 2.42 - B & 2.63 - E & 2.84 - F & 2.105 - B & 2.126 - C \\
\hline
2.127 - C & 2.159 - D & 2.191 - F & 2.223 - A & 2.255 - D & 2.287 - C \\
\hline
2.128 - E & 2.160 - E & 2.192 - D & 2.224 - C & 2.256 - C & 2.288 - B \\
\hline
2.129 - B & 2.161 - F & 2.193 - A & 2.225 - E & 2.257 - C & 2.289 - A \\
\hline
2.130 - B & 2.162 - E & 2.194 - B & 2.226 - A & 2.258 - D & 2.290 - A \\
\hline
2.131 - E & 2.163 - A & 2.195 - D & 2.227 - F & 2.259 - E & 2.291 - A \\
\hline
2.132 - D & 2.164 - F & 2.196 - F & 2.228 - D & 2.260 - B & 2.292 - E \\
\hline
2.133 - D & 2.165 - E & 2.197 - E & 2.229 - A & 2.261 - D & 2.293 - A \\
\hline
2.134 - A & 2.166 - A & 2.198 - D & 2.230 - E & 2.262 - E & 2.294 - E \\
\hline
2.135 - D & 2.167 - A & 2.199 - A & 2.231 - B & 2.263 - E & 2.295 - B \\
\hline
2.136 - A & 2.168 - C & 2.200 - B & 2.232 - C & 2.264 - E & 2.296 - F \\
\hline
2.137 - E & 2.169 - D & 2.201 - B & 2.233 - D & 2.265 - E & 2.297 - C \\
\hline
2.138 - D & 2.170 - A & 2.202 - A & 2.234 - F & 2.266 - F & 2.298 - D \\
\hline
2.139 - E & 2.171 - B & 2.203 - C & 2.235 - E & 2.267 - D & 2.299 - B \\
\hline
2.140 - E & 2.172 - C & 2.204 - E & 2.236 - C & 2.268 - C & 2.300 - C \\
\hline
2.141 - D & 2.173 - B & 2.205 - A & 2.237 - B & 2.269 - F & 2.301 - F \\
\hline
2.142 - D & 2.174 - C & 2.206 - B & 2.238 - D & 2.270 - F & 2.302 - A \\
\hline
2.143 - D & 2.175 - F & 2.207 - C & 2.239 - A & 2.271 - D & 2.303 - C \\
\hline
2.144 - C & 2.176 - E & 2.208 - E & 2.240 - A & 2.272 - E & 2.304 - A \\
\hline
2.145 - C & 2.177 - D & 2.209 - C & 2.241 - C & 2.273 - D & 2.305 - A \\
\hline
2.146 - A & 2.178 - F & 2.210 - B & 2.242 - D & 2.274 - D & 2.306 - A \\
\hline
2.147 - A & 2.179 - D & 2.211 - B & 2.243 - E & 2.275 - A & 2.307 - C \\
\hline
2.148 - C & 2.180 - A & 2.212 - A & 2.244 - F & 2.276 - D & 2.308 - A \\
\hline
2.149 - D & 2.181 - C & 2.213 - D & 2.245 - A & 2.277 - C & 2.309 - C \\
\hline
2.150 - C & 2.182 - E & 2.214 - A & 2.246 - C & 2.278 - E & 2.310 - A \\
\hline
2.151 - A & 2.183 - B & 2.215 - A & 2.247 - D & 2.279 - A & 2.311 - A \\
\hline
2.152 - B & 2.184 - A & 2.216 - C & 2.248 - E & 2.280 - D & 2.312 - A \\
\hline
2.153 - B & 2.185 - D & 2.217 - D & 2.249 - B & 2.281 - D & 2.313 - A \\
\hline
2.154 - D & 2.186 - B & 2.218 - C & 2.250 - C & 2.282 - C & 2.314 - E \\
\hline
2.155 - D & 2.187 - B & 2.219 - E & 2.251 - E & 2.283 - B &  \\
\hline
2.156 - C & 2.188 - A & 2.220 - C & 2.252 - D & 2.284 - E &  \\
\hline
2.157 - C & 2.189 - C & 2.221 - D & 2.253 - B & 2.285 - B &  \\
\hline
2.158 - C & 2.190 - E & 2.222 - C & 2.254 - B & 2.286 - B &  \\
\hline
\end{tabular}
\end{center}

%% End %%