% iulie 2006, Admitere UPB, Fizică FA. Enunțuri şi rezolvare %

2006.A.1. Produsul dintre temperatura şi densitatea unui gaz ideal este constant în transformarea: ( 6 pct.)\\ a) generală; b) în nici un fel de transformare; c) izotermă; d) izocoră; e) adiabată; f) izobară.\\ Din expresia densității gazului ideal, $\rho=\frac{m}{V}=\frac{p \mu}{R T}$, din condiția ca produsul $\rho T=\frac{p \mu}{R}=$ constant rezultă $p=$ constant, adică o transformare izobară. Răspuns corect f.\\

2006.A.2. Un conductor cu rezistenţa de $5 \Omega$ este parcurs în 50 de secunde de un număr de $2 \cdot 10^{21}$ electroni. Tensiunea electrică între capetele conductorului este: (sarcina electronului este de $1,6 \cdot 10^{-19} \mathrm{C}$) (6 pct.)\\ a) $18 \mathrm{~V}$; b) $2 \mathrm{~V}$; c) $32 \mathrm{~V}$; d) $20 \mathrm{~V}$; e) $40 \mathrm{~V}$; f) $0,2 \mathrm{~V}$.\\ Intensitatea curentului electric este $I=\frac{N e}{t}=\frac{U}{R}$, de unde tensiunea electrică la capetele conductorului este egală cu $U=\frac{N e R}{t}=32 \mathrm{~V}$. Răspuns corect c.\\

2006.A.3. Un corp este ridicat vertical în sus cu ajutorul unui fir. Cu ce accelerație trebuie efectuată ridicarea, pentru ca tensiunea din fir să fie egală cu greutatea corpului? ( 6 pct.)\\ a) $56 \mathrm{~m} / \mathrm{s}^{2}$; b) $1 \mathrm{~m} / \mathrm{s}^{2}$; c) $10 \mathrm{~m} / \mathrm{s}^{2}$; d) $0 \mathrm{~m} / \mathrm{s}^{2}$; e) $9,8 \mathrm{~m} / \mathrm{s}^{2}$; f) $g$.\\ La urcarea corpului cu accelerația $a$, conform condiției problemei, tensiunea din fir este egală cu $T=m(g+a)=m g$. Prin urmare, $a=0$. Răspuns corect d.\\

2006.A.4. O cantitate de gaz ideal monoatomic ($C_{v}=3 R / 2$) se destinde izobar. Raportul dintre lucrul mecanic efectuat şi căldura primită în timpul acestui proces este: ( 8 pct.)\\ a) $50 \%$; b) $40 \%$; c) $1 \%$; d) $12 \%$; e) $77 \%$; f) $41 \%$.\\ Într-o transformare izobară, raportul $\frac{L}{Q}=\frac{\nu R \Delta T}{\nu C_{p} \Delta T}=\frac{R}{C_{p}}=\frac{R}{R+C_{V}}=\frac{2}{5}=40 \%$. Răspuns corect b.\\

2006.A.5. Se leagă $n$ rezistențe identice, mai întâi în serie și apoi în paralel. Raportul dintre rezistențele echivalente în cele două cazuri este: ( 8 pct.)\\ a) $\frac{n}{(n+1)^{2}}$; b) $\frac{n^{2}}{n+1}$; c) $n$; d) $n^{2}$; e) $1 / n$; f) $1$.\\ La legarea în serie a rezistoarelor, $R_{s}=n R$, iar la legarea în paralel, $R_{p}=\frac{R}{n}$. Prin urmare, raportul $\frac{R_{s}}{R_{p}}=n^{2}$. Răspuns corect d.\\

2006.A.6. Spațiul total străbătut de un corp în cădere liberă, care în ultimele 4 secunde parcurge $400 \mathrm{~m}$, este ($g=10 \mathrm{~m} / \mathrm{s}^{2}$): ( 8 pct.)\\ a) $815 \mathrm{~m}$; b) $370 \mathrm{~m}$; c) $700 \mathrm{~m}$; d) $750 \mathrm{~m}$; e) $740 \mathrm{~m}$; f) $720 \mathrm{~m}$.\\ Spațiul parcurs de corp în ultimele $t_{1}=4 \mathrm{~s}$ este egal cu:\\ $h=g\left(t-t_{1}\right) t_{1}+\frac{1}{2} g t_{1}^{2}=g t t_{1}-\frac{1}{2} g t_{1}^{2}$, unde $t$ este durata totală de cădere a corpului. Astfel, $t=\frac{2 h+g t_{1}^{2}}{2 g t_{1}}=12 \mathrm{~s}$.\\ Spațiul total de cădere este egal cu $H=\frac{1}{2} g t^{2}=720 \mathrm{~m}$. Răspuns corect f.\\

2006.A.7. Două fire conductoare drepte, paralele şi foarte lungi se află în aer la distanța de $20 \mathrm{~cm}$ unul de altul. Firele sunt parcurse în acelaşi sens de curenţi egali cu $2 \mathrm{~A}$. Inducţia magnetică la jumătatea distanței dintre fire este (permeabilitatea magnetică a aerului este $\mu_{0}=4 \pi 10^{-7} \mathrm{~N} / \mathrm{A}^{2}$): ( 4 pct.)\\ a) $8 \cdot 10^{-6} \mathrm{~T}$; b) $8 \pi \cdot 10^{-6} \mathrm{~T}$; c) $4 \pi \cdot 10^{-6} \mathrm{~T}$; d) $2 \pi \cdot 10^{-6} \mathrm{~T}$; e) $0 \mathrm{~T}$; f) $4 \cdot 10^{-6} \mathrm{~T}$.\\ La jumătatea distanței dintre conductori, vectorii inducție ai câmpului magnetic produs de cei doi curenți electrici paraleli, egali şi de acelaşi semn sunt egali şi de sens opus, astfel că inducția magnetică totală este nulă. Răspuns corect e.\\

2006.A.8. Randamentul unei maşini termice ideale care funcționează după un ciclu Carnot este de $20 \%$. Cât devine randamentul dacă temperatura sursei calde creşte de două ori, iar cea a sursei reci rămâne constantă? (4 pct.)\\ a) $50 \%$; b) $60 \%$; c) $90 \%$; d) $40 \%$; e) $80 \%$; f) $55 \%$.\\ Randamentul maşinii termice ideale în cele două cazuri are valoarea, $\eta_{1}=1-\frac{T_{2}}{T_{1}}$ şi respectiv $\eta_{2}=1-\frac{T_{2}}{2 T_{1}}$, de unde $\eta_{2}=\frac{1+\eta_{1}}{2}=0,6=60 \%$. Răspuns corect b.\\

2006.A.9. Se consideră un circuit format dintr-un rezistor legat la o sursă cu t.e.m. $E=2 \mathrm{~V}$ și cu rezistența internă nenulă. Care este tensiunea pe rezistor ştiind că puterea disipată pe acesta este maximă? (4 pct.)\\ a) nu se poate calcula; b) $5 \mathrm{~V}$; c) $1 \mathrm{~V}$; d) $1,5 \mathrm{~V}$; e) $0,1 \mathrm{~V}$; f) $2 \mathrm{~V}$.\\ Dacă puterea disipată pe rezistența exterioară este maximă, atunci $R=r$ şi tensiunea la borne $U=I R=\frac{E}{2}=1 \mathrm{~V}$, unde intensitatea curentului $I=\frac{E}{2 R}$. Răspuns corect c.\\

2006.A.10. În SI fluxul magnetic se măsoară în: (4 pct.)\\ a) $\mathrm{V} / \mathrm{m}$; b) $\mathrm{Wb}$; c) $\mathrm{N} / \mathrm{m}$; d) $\mathrm{N} / \mathrm{A}^{2}$; e) $\mathrm{T}$; f) $\mathrm{A} / \mathrm{m}$.\\ Fluxul magnetic se măsoară în $\mathrm{Wb}$. Răspuns corect b.\\

2006.A.11. Un motociclist se deplasează între două localităţi astfel: pe prima jumătate a distanței cu viteza $v_{1}=20 \mathrm{~m} / \mathrm{s}$, iar pe cealaltă jumătate cu viteza $v_{2}=30 \mathrm{~m} / \mathrm{s}$. Viteza medie a motociclistului pe întreaga distanţă este: (4 pct.)\\ a) $100 \mathrm{~km} / \mathrm{h}$; b) $15 \mathrm{~m} / \mathrm{s}$; c) $18 \mathrm{~m} / \mathrm{s}$; d) $25 \mathrm{~m} / \mathrm{s}$; e) $24 \mathrm{~m} / \mathrm{s}$; f) $70 \mathrm{~km} / \mathrm{h}$.\\ Conform definiției vitezei medii, $v_{m}=\frac{d}{t_{1}+t_{2}}=\frac{d}{\frac{d}{2 v_{1}}+\frac{d}{2 v_{2}}}=\frac{2 v_{1} v_{2}}{v_{1}+v_{2}}=24 \mathrm{~m} / \mathrm{s}$. Răspuns corect e.\\

2006.A.12. Expresia energiei cinetice pentru un corp de masă $m$ şi viteză $v$ este: (4 pct.)\\ a) $\frac{m v^{2}}{2}$; b) $m v^{2}$; c) $\frac{p^{2}}{m}$; d) $m v$; e) $m g v$; f) $m g h$.\\ Expresia energiei cinetice pentru un corp de masă $m$ și viteză $v$ este $E_{c}=\frac{1}{2} m v^{2}$. Răspuns corect a.\\

2006.A.13. Un corp atârnat de un resort cu constanta elastică $10 \mathrm{~N} / \mathrm{m}$ produce alungirea $x_{1}$. Acelaşi corp, atârnat de un resort cu constanta elastică $50 \mathrm{~N} / \mathrm{m}$ produce alungirea $x_{2}$. Raportul $x_{2} / x_{1}$ este: ( 4 pct.)\\ a) $4$; b) $0,15$; c) nu se poate calcula; d) $1 / 50$; e) $1 / 5$; f) $0,25$.\\ Din condițiile $m g=k_{1} x_{1}$ şi respectiv $m g=k_{2} x_{2}$ rezultă că $\frac{x_{2}}{x_{1}}=\frac{k_{1}}{k_{2}}=\frac{1}{5}$. Răspuns corect e.\\

2006.A.14. Un cilindru vertical închis la capete este separat în două compartimente printr-un piston mobil de volum neglijabil. În cele două compartimente se află mase egale din acelaşi gaz ideal, la aceeaşi temperatură $T_{1}$. La echilibru, raportul volumelor celor două compartimente este $n=3$. Care va fi raportul volumelor, dacă temperatura comună creşte la $4 T_{1} / 3$ ? (4 pct.)\\ a) $\sqrt{2} / 3$; b) $\sqrt{2}-1$; c) $\sqrt{2}$; d) $\sqrt{2}+1$; e) nu se poate calcula; f) $2$.\\ În cele două cazuri, diferența între presiunile celor două mase de gaz din compartimentul de jos şi cel de sus este egală cu presiunea exercitată de greutatea pistonului mobil, adică $p_{2}-p_{1}=p_{2}^{\prime}-p_{1}^{\prime}$, unde conform ecuației termice de stare, $p_{1}=\frac{m R T_{1}}{n V}$, $p_{2}=\frac{m R T_{1}}{V}$, $p_{1}^{\prime}=\frac{4 m R T_{1}}{3 x V_{1}}$ şi $p_{2}^{\prime}=\frac{4 m R T_{1}}{3 V_{1}}$. Astfel, $\frac{m R T_{1}}{V}-\frac{m R T_{1}}{n V}=\frac{4 m R T_{1}}{3 V_{1}}-\frac{4 m R T_{1}}{3 x V_{1}}$. Volumul total ocupat de gaz în cele două cazuri este acelaşi, adică $(n+1) V=(x+1) V_{1}$. Eliminând raportul $\frac{V}{V_{1}}$ între cele două relații rezultă ecuația $x^{2}-2 x-1=0$, care are rădăcina pozitivă $x=1+\sqrt{2}$. Răspuns corect d.\\

2006.A.15. Un fir de nichelină cu rezistivitatea $\rho=4 \cdot 10^{-7} \Omega \cdot \mathrm{~m}$ și secţiunea transversală $S=5 \cdot 10^{-7} \mathrm{~m}^{2}$ este parcurs de un curent cu intensitate $I=0,2 \mathrm{~A}$ atunci când la capetele lui se aplică o tensiune de $3,2 \mathrm{~V}$. Lungimea conductorului este: ( $\mathbf{4}$ pct.)\\ a) $22 \mathrm{~m}$; b) $23 \mathrm{~m}$; c) $18 \mathrm{~m}$; d) $30 \mathrm{~m}$; e) $21 \mathrm{~m}$; f) $20 \mathrm{~m}$.\\ Conform legii lui Ohm, intensitatea curentului prin fir $I=\frac{U}{R}=\frac{U S}{\rho l}$, de unde lungimea firului, $l=\frac{U S}{\rho I}=20 \mathrm{~m}$. Răspuns corect f.\\

2006.A.16. Lucrul mecanic este o mărime fizică ( 4 pct.)\\ a) scalară şi se măsoară în $\mathrm{W}$; b) vectorială şi se măsoară în $\mathrm{W}$; c) scalară şi se măsoară în $\mathrm{N}$; d) vectorială şi se măsoară în C.P.; e) vectorială şi se măsoară în $\mathrm{J}$; f) scalară şi se măsoară în $\mathrm{J}$.\\ Lucrul mecanic este o mărime fizică scalară şi se măsoară în $\mathrm{J}$. Răspuns corect f.\\

2006.A.17. În ce transformare a gazului ideal, variaţia energiei interne este nulă? (4 pct.)\\ a) Adiabată; b) Izobară; c) Izotermă; d) Generală; e) Nu există o astfel de transformare; f) Izocoră.\\ Din condiția ca variația energiei interne să fie nulă, adică $\Delta U=\nu C_{V} \Delta T=0$, rezultă că $\Delta T=0$, adică $T=$ constant, ceea ce se întâmplă într-o transformare izotermă. Răspuns corect c.\\

2006.A.18. În SI unitatea de măsură pentru căldura specifică la presiune constantă este: (4 pct.)\\ a) $\frac{\mathrm{J}}{\mathrm{kg} \cdot \mathrm{K}}$; b) $\frac{\mathrm{J}}{\mathrm{K}}$; c) $\frac{\mathrm{J}}{\mathrm{kg}}$; d) $\frac{\mathrm{Pa}}{\mathrm{kg} \cdot \mathrm{K}}$; e) $\frac{\mathrm{J}}{\mathrm{Pa}}$; f) $\frac{\mathrm{J}}{\mathrm{mol} \cdot \mathrm{K}}$.\\ Conform definiției, căldura specifică este egală cu $c=\frac{Q}{m \Delta T}$ şi se măsoară în $\frac{\mathrm{J}}{\mathrm{kg} \mathrm{K}}$. Răspuns corect a.\\
