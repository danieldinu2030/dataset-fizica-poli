% iulie 2002, Admitere UPB, Fizică FA. Enunțuri şi rezolvare %

2002.A.1. Considerând $R \approx 8,32 \mathrm{~J} / \mathrm{mol} \cdot \mathrm{K}$, căldura specifică la volum constant a unui gaz ideal cu $\mu=32 \cdot 10^{-3} \mathrm{~kg} / \mathrm{mol}$ şi $\gamma=1,4$, este: ( 6 pct.)\\ a) $100 \mathrm{~J} / \mathrm{kg} \cdot \mathrm{K}$; b) $600 \mathrm{~J} / \mathrm{kg} \cdot \mathrm{K}$; c) $650 \mathrm{~J} / \mathrm{kg} \cdot \mathrm{K}$; d) $500 \mathrm{~J} / \mathrm{kg} \cdot \mathrm{K}$; e) $700 \mathrm{~J} / \mathrm{kg} \cdot \mathrm{K}$; f) $800 \mathrm{~J} / \mathrm{kg} \cdot \mathrm{K}$.\\ Căldura specifică la volum constant a unui gaz ideal se poate scrie sub forma $c_{V}=\frac{C_{V}}{\mu}=\frac{R}{\mu(\gamma-1)}=650 \mathrm{~J} / \mathrm{kgK}$. Răspuns corect c.\\

2002.A.2. Un circuit oscilant format dintr-o bobină şi un condensator variabil este acordat pe lungimea de undă $\lambda_{0}$. Mărind de 4 ori capacitatea condensatorului, lungimea de undă la noua rezonanță devine: ( 6 pct.)\\ a) $4 \cdot \lambda_{0}$; b) $2 \cdot \lambda_{0}$; c) $\lambda_{0} / 2$; d) $\lambda_{0}$; e) $0,2 \cdot \lambda_{0}$; f) $\lambda_{0} / 4$.\\ Lungimea de undă la rezonanță a circuitului oscilant este $\lambda_{0}=\nu T=2 \pi \nu \sqrt{L C}$, iar în cazul măririi capacității condensatorului, $\lambda_{1}=2 \pi \nu \sqrt{4 L C}=2 \lambda_{0}$. Răspuns corect b.\\

2002.A.3. Mişcarea unui corp este descrisă de ecuația $x=-2 t^{2}+4 t+4$, cu $x$ şi $t$ măsurându-se în unități SI. Viteza medie a mişcării corpului în intervalul $1 \mathrm{~s} \leq t \leq 5 \mathrm{~s}$, este: ( 6 pct.)\\ a) $-8 \mathrm{~m} / \mathrm{s}$; b) $+9 \mathrm{~m} / \mathrm{s}$; c) $+12 \mathrm{~m} / \mathrm{s}$; d) $-10,5 \mathrm{~m} / \mathrm{s}$; e) $-4 \mathrm{~m} / \mathrm{s}$; f) $-3,5 \mathrm{~m} / \mathrm{s}$.\\ Poziția corpului la momentul $t_{1}=1 \mathrm{~s}$ este $x_{1}=6 \mathrm{~m}$, iar la momentul $t_{2}=5 \mathrm{~s}$ este dată de coordonata $x_{2}=-26 \mathrm{~m}$.\\ Viteza medie este egală cu $v_{m}=\frac{x_{2}-x_{1}}{t_{2}-t_{1}}=-8 \mathrm{~m} / \mathrm{s}$. Răspuns corect a.\\

2002.A.4. Intensitatea curentului alternativ care străbate un circuit serie RLC este $I_{1}=3 \mathrm{~A}$. Dacă rezistența $R$ se consideră nulă, intensitatea curentului prin circuit devine $I_{2}=5 \mathrm{~A}$. Intensitatea curentului prin circuitul RLC, aflat la rezonanță, este: (8 pct.)\\ a) $3,75 \mathrm{~A}$; b) $4 \mathrm{~A}$; c) $8 \mathrm{~A}$; d) $3 \mathrm{~A}$; e) $3,5 \mathrm{~A}$; f) $5,83 \mathrm{~A}$.\\ Din expresiile intensității curentului în cele două cazuri, $I_{1}=\frac{U}{\sqrt{R^{2}+\left(\omega L-\frac{1}{\omega C}\right)^{2}}}$ şi respectiv $I_{2}=\frac{U}{\omega L-\frac{1}{\omega C}}$, rezultă expresia rezistenței electrice, $R=U \sqrt{\frac{1}{I_{1}^{2}}-\sqrt{\frac{1}{I_{2}^{2}}}}$, astfel că valoarea intensității curentului la rezonanță este egală cu:\\ $I_{\text {rez}}=\frac{U}{R}=\frac{1}{\sqrt{\frac{1}{I_{1}^{2}}-\sqrt{\frac{1}{I_{2}^{2}}}}}=3,75 \mathrm{~A}$. Răspuns corect a.\\

2002.A.5. Un gaz ideal cu exponentul adiabatic $\gamma$ efectuează transformarea descrisă de ecuația $p=\alpha V$, $\alpha$ fiind o constantă, între două stări oarecare (1) şi (2). Căldura molară a gazului în această transformare este: (8 pct.)\\ a) $\frac{2 R(\gamma-1)}{\gamma+1}$; b) $\frac{2 R}{\gamma-1}$; c) $\frac{3 R}{2}$; d) $R \gamma$; e) $\frac{R(\gamma+1)}{2(\gamma-1)}$; f) $\frac{R \gamma}{\gamma-1}$.\\ Ecuația transformării se poate scrie sub forma unei politrope, $p V^{-1}=$ const., a cărui indice este $n=-1=\frac{C-C_{p}}{C-C_{V}}$, de unde căldura molară la gazului în această transformare este egală cu:\\ $C=\frac{1}{2}\left(C_{V}+C_{p}\right)=\frac{1}{2}\left(\frac{R}{\gamma-1}+\frac{\gamma R}{\gamma-1}\right)=\frac{R(\gamma+1)}{2(\gamma-1)}$. Răspuns corect e.\\

2002.A.6. Un punct material oscilează după legea $y=A \sin (\pi t+\pi / 4)$ (în m). Raportul dintre energiile cinetică şi potențială ale punctului material la momentul $t_{1}=T / 4$ de la pornire, este: ( 8 pct.)\\ a) $0,5$; b) $4$; c) $3$; d) $0,1$; e) $1$; f) $0$.\\ Raportul energiilor cinetică şi potențială la momentul $t_{1}=\frac{T}{4}=\frac{1}{2} \mathrm{~s}$ este egal cu $\frac{E_{c}}{E_{p}}=\operatorname{tg}^{2}\left(\frac{\pi}{2}+\frac{\pi}{4}\right)=1$, unde perioada $T=\frac{2 \pi}{\omega}=2 \mathrm{~s}$. Răspuns corect e.\\

2002.A.7. Dacă energia unui condensator plan încărcat este $C U^{2} / 2$, densitatea de energie a câmpului electric în dielectricul dintre armăturile condensatorului este: ( 4 pct.)\\ a) $E^{2} / 2 \varepsilon$; b) $\mu H^{2} / 2$; c) $\varepsilon E^{2}$; d) $\varepsilon_{0} E / 2 \varepsilon$; e) $\varepsilon E / H$; f) $\varepsilon E^{2} / 2$.\\ Densitatea de energie electrică înmagazinată în dielectricul condensatorului este $w_{el}=\frac{W_{el}}{V}=\frac{\varepsilon S U^{2}}{2 d^{2} S}=\frac{1}{2} \varepsilon E^{2}$. Răspuns corect f.\\

2002.A.8. Masa molară a amestecului format din $60 \mathrm{~g}$ de hidrogen ($\mu_{H_{2}}=2 \cdot 10^{-3} \mathrm{~kg} / \mathrm{mol}$) şi $120 \mathrm{~g}$ de dioxid de carbon ($\mu_{\mathrm{CO}_{2}}=44 \cdot 10^{-3} \mathrm{~kg} / \mathrm{mol}$) este: ( 4 pct.)\\ a) $5 \cdot 10^{-3} \mathrm{~kg} / \mathrm{mol}$; b) $8 \cdot 10^{-4} \mathrm{~kg} / \mathrm{mol}$; c) $6 \cdot 10^{-3} \mathrm{~kg} / \mathrm{mol}$; d) $5,2 \cdot 10^{-3} \mathrm{~kg} / \mathrm{mol}$; e) $11 \cdot 10^{-3} \mathrm{~kg} / \mathrm{mol}$; f) $5,5 \cdot 10^{-3} \mathrm{~kg} / \mathrm{mol}$.\\ Masa molară a amestecului de gaze este egală cu:\\ $\mu_{am}=\frac{m_{1}+m_{2}}{v_{1}+v_{2}}=\frac{m_{1}+m_{2}}{\frac{m_{1}}{\mu_{1}}+\frac{m_{2}}{\mu_{2}}}=\frac{\mu_{1} \mu_{2}\left(m_{1}+m_{2}\right)}{m_{1} \mu_{2}+m_{2} \mu_{1}}=5,5 \cdot 10^{-3} \mathrm{~kg} / \mathrm{mol}$.\\ Răspuns corect f.\\

2002.A.9. Trei baterii identice, legate în serie, alimentează un rezistor cu rezistența de $60 \Omega$. Dacă se scurtcircuitează una dintre baterii, intensitatea curentului electric scade de 1,4 ori. Rezistența internă a fiecărei baterii, este: (4 pct.)\\ a) $15 \Omega$; b) $10 \Omega$; c) $1 \Omega$; d) $5 \Omega$; e) $16 \Omega$; f) $6 \Omega$.\\ Din legea lui Ohm în cele două cazuri, $I_{1}=\frac{3 E}{R+3 r}$ şi respectiv $I_{2}=\frac{I_{1}}{1,4}=\frac{2 E}{R+2 r}$ rezultă pentru rezistența internă valoarea $r=5 \Omega$. Răspuns corect d.\\

2002.A.10. În cazul unui motor care funcționează după un ciclu Carnot și absoarbe într-un ciclu căldura $Q_{1}=2500 \mathrm{~J}$ de la sursa caldă a cărei temperatură este $t_{1}=227^{\circ} \mathrm{C}$, temperatura sursei reci fiind $t_{2}=27^{\circ} \mathrm{C}$, căldura $\left|Q_{2}\right|$ cedată sursei reci, este: (4 pct.)\\ a) $1200 \mathrm{~J}$; b) $1 \mathrm{~kJ}$; c) $500 \mathrm{~J}$; d) $1500 \mathrm{~J}$; e) $0,4 \mathrm{~MJ}$; f) $2000 \mathrm{~J}$.\\ Într-un ciclu Carnot, $\frac{Q_{1}}{\left|Q_{2}\right|}=\frac{T_{1}}{T_{2}}$, de unde $\left|Q_{2}\right|=1500 \mathrm{~J}$. Răspuns corect d.\\

2002.A.11. Unitatea de măsură pentru presiune, în SI, este: ( 4 pct.)\\ a) $\mathrm{Pa}$; b) $N \cdot m$; c) $\mathrm{N} / \mathrm{m}$; d) $\mathrm{atm}$; e) $\mathrm{torr}$; f) $\mathrm{at}$.\\ Răspuns corect a.\\

2002.A.12. Un consumator constă din 20 rezistori cu rezistența $R_{1}=40 \Omega$ fiecare şi 100 rezistori cu $R_{2}=200 \Omega$ fiecare. Rezistorii fiind legați toți în paralel, consumatorul are rezistență electrică totală: ( 4 pct.)\\ a) $500 \Omega$; b) $4 \Omega$; c) $0,5 \Omega$; d) $1 \Omega$; e) $240 \Omega$; f) $3 \Omega$.\\ La legarea în paralel a rezistoarelor, $\frac{1}{R_{ech}}=\frac{n_{1}}{R_{1}}+\frac{n_{2}}{R_{2}}=1$, de unde $R_{\text {ech}}=1 \Omega$. Răspuns corect d.\\

2002.A.13. Unitatea de măsură pentru capacitatea calorică, în SI, este: (4 pct.)\\ a) $\mathrm{J} / \mathrm{K}^{2}$; b) $\mathrm{J}^{2} / \mathrm{K}$; c) $\mathrm{J} \cdot \mathrm{K}$; d) $\mathrm{J}$; e) $\mathrm{J} / \mathrm{kg}$; f) $\mathrm{J} / \mathrm{K}$.\\ Răspuns corect f.\\

2002.A.14. Un corp are energia cinetică $E_{c}=20 \mathrm{~J}$. Lucrul mecanic efectuat asupra corpului pentru a-i tripla impulsul, este: ( $\mathbf{4}$ pct.)\\ a) $60 \mathrm{~J}$; b) $180 \mathrm{~J}$; c) $40 \mathrm{~J}$; d) $2,5 \mathrm{~J}$; e) $160 \mathrm{~J}$; f) $160 \mathrm{~W}$.\\ Conform teoremei de variație a energiei cinetice:\\ $L=\Delta E_{c}=\frac{\left(3 p_{1}\right)^{2}}{2} \mathrm{~m}-\frac{p_{1}^{2}}{2} \mathrm{~m}=8 E_{c 1}=160 \mathrm{~J}$, unde am utilizat pentru energia cinetică relația $E_{c}=\frac{1}{2} m v^{2}=\frac{m^{2} v^{2}}{2} \mathrm{~m}=\frac{p^{2}}{2} \mathrm{~m}$. Răspuns corect e.\\

2002.A.15. Ecuația $V T^{n}=$ const. descrie un proces termodinamic izobar, dacă: (4 pct.)\\ a) $n=-1$; b) $n=\frac{\gamma-1}{\gamma}$; c) $n=\gamma-1$; d) $n=1$; e) $n=\gamma$; f) $n=0$.\\ Transformare izobară se poate scrie sub forma $V T^{-1}=$ const., de unde $n=-1$. Răspuns corect a.\\

2002.A.16. Un corp legat de un resort cu constanta elastică de $0,8 \pi^{2} \mathrm{~N} / \mathrm{m}$, oscilând cu perioada de $1 \mathrm{~s}$, are masa: (4 pct.)\\ a) $0,2 \mathrm{~t}$; b) $1 \mathrm{~g}$; c) $5 \mathrm{~kg}$; d) $0,2 \mathrm{~kg}$; e) $0,15 \mathrm{~kg}$; f) $1 \mathrm{~kg}$.\\ Din expresia perioadei oscilatorului, $T=2 \pi \sqrt{\frac{m}{k}}$, rezultă masa acestuia:\\ $m=\frac{k T^{2}}{4 \pi^{2}}=0,2 \mathrm{~kg}$. Răspuns corect d.\\

2002.A.17. Un corp este lansat în sus pe un plan înclinat cu unghiul $\alpha$, pe care se mişcă cu frecare (coeficientul de frecare fiind $\mu$). După oprire, corpul nu va porni înapoi spre baza planului dacă este satisfăcută condiția: (4 pct.)\\ a) $\mu \geq \sin \alpha$; b) $\mu \geq \operatorname{tg}^{2} \alpha$; c) $\mu \geq \frac{1-\sin \alpha}{\cos \alpha}$; d) $\mu \geq \sin ^{2} \alpha$; e) $\mu \geq \operatorname{tg} \alpha$; f) $\mu \geq 0,3$.\\ Din condiția ca $G_{\tan} \leq F_{frec}$ rezultă $\mu \geq \operatorname{tg} \alpha$. Răspuns corect e.\\

2002.A.18. Unitatea de măsură pentru inductanță, în SI, este: (4 pct.)\\ a) $\mathrm{C} / \mathrm{m}$; b) $\mathrm{J}$; c) $\mathrm{H} / \mathrm{m}$; d) $\mathrm{H}$; e) $\mathrm{W}$; f) $\mathrm{H} \cdot \mathrm{m}$.\\ Răspuns corect d.\\
