\section{Admitere F iulie 2007}

2007.A.1. Rezistența circuitului exterior al unei baterii cu tensiunea electromotoare de $1,5 \mathrm{~V}$ este de $2 \Omega$. Dacă tensiunea la bornele bateriei este de $1 \mathrm{~V}$ atunci valoarea rezistenței sale interne este: ( 4 pct.)\\ a) $4 \Omega$; b) $0,1 \Omega$; c) $1,1 \Omega$; d) $0,5 \Omega$; e) $1 \Omega$; f) $2 \Omega$.\\ Din expresia tensiunii la borne, $U=\frac{E R}{R+r}$, rezistența internă a sursei este $r=\frac{R(E-U)}{U}=1 \Omega$. Răspuns corect e.\\

2007.A.2. Un gaz ideal are indicele adiabatic $1,4$. Căldurile molare la volum şi respectiv presiune constantă sunt: (4 pct.)\\ a) $\frac{3}{2} R$, $\frac{5}{2} R$; b) $\frac{2}{5} R$, $\frac{2}{7} R$; c) $\frac{2}{3} R$, $\frac{2}{5} R$; d) $\frac{1}{3} R$, $\frac{1}{4} R$; e) $\frac{5}{2} R$, $\frac{7}{2} R$; f) $3 R$, $4 R$.\\ $C_{V}=\frac{R}{\gamma-1}=\frac{5}{2} R$ şi $C_{p}=\frac{\gamma R}{\gamma-1}=\frac{7}{2} R$. Răspuns corect e.\\

2007.A.3. Expresia forței de interacţiune dintre două conductoare paralele, foarte lungi, parcurse de curenți electrici (forţa electrodinamică) este: (4 pct.)\\ a) $F=\mu \frac{I_{1} l}{2 \pi d I_{2}}$; b) $F=\mu \frac{I_{1} I_{2} l}{d}$; c) $F=\mu \frac{I_{1} I_{2} l}{4 \pi d}$; d) $F=\mu \frac{I_{1} I_{2} l}{2 \pi \sqrt{d}}$; e) $F=\mu \frac{I_{1} I_{2} l}{2 \pi d}$;  f) $F=\mu \frac{2 \pi d}{I_{1} I_{2} l}$.\\ Răspuns corect e.\\

2007.A.4. Atunci când un vehicul se deplasează cu viteza constantă de $15 \mathrm{~km} / \mathrm{h}$, motorul său dezvoltă o putere de $15 \mathrm{~kW}$. Forța de rezistență pe care o întâmpină vehiculul este: ( 4 pct.)\\ a) $300 \mathrm{~N}$; b) $360 \mathrm{~N}$; c) $250 \mathrm{~N}$; d) $3,6 \mathrm{~kN}$; e) $1,2 \mathrm{~kN}$; f) $100 \mathrm{~N}$.\\ Viteza fiind constantă, $F_{\text {trac}}=F_{\text {rez}}=\frac{P}{v}$, de unde $F_{\text {rez}}=3,6 \cdot 10^{3} \mathrm{~N}$. Răspuns corect d.\\

2007.A.5. Un conductor de lungime $100 \mathrm{~m}$ și diametru $1 \mathrm{~mm}$ are rezistenţa electrică de $56 \Omega$. Rezistivitatea materialului din care este confectionat conductorul este: ( 4 pct.)\\ a) $3 \pi \cdot 10^{-8} \Omega \cdot \mathrm{~m}$; b) $14 \pi \cdot 10^{-7} \Omega \cdot \mathrm{~m}$; c) $2 \pi \cdot 10^{-8} \Omega \cdot \mathrm{~m}$; d) $14 \pi \cdot 10^{-8} \Omega \cdot \mathrm{~m}$; e) $7 \pi \cdot 10^{-7} \Omega \cdot \mathrm{~m}$; f) $4 \pi \cdot 10^{-8} \Omega \cdot \mathrm{~m}$.\\ Din expresia rezistenței electrice, $R=\rho \frac{l}{S}$, rezistivitatea este:\\ $\rho=14 \pi \cdot 10^{-8} \Omega \mathrm{~m}=43,98 \cdot 10^{-8} \Omega \mathrm{~m}$. Răspuns corect d.\\

2007.A.6. Prin încălzirea cu $10 \mathrm{~K}$ a unui gaz ideal închis într-un recipient de volum constant, presiunea sa creşte de 10 ori. Temperatura inițială a gazului este: ( 4 pct.)\\ a) $11 \mathrm{~K}$; b) $\frac{27}{13} \mathrm{~K}$; c) $\frac{10}{7} \mathrm{~K}$; d) $117 \mathrm{~K}$; e) $\frac{10}{9} \mathrm{~K}$; f) $\frac{8}{3} \mathrm{~K}$.\\ Din ecuația transformării izocore, $\frac{p}{T}=\frac{10 p}{T+\Delta T}$, rezultă $T=\frac{10}{9} \mathrm{~K}$. Răspuns corect e.\\

2007.A.7. Un mol de gaz ideal monoatomic se destinde după legea $T V^{-2}=a$ ($a$ este o constantă pozitivă). Căldura sa molară în timpul acestui proces are valoarea: (8 pct.)\\ a) $0,5 R$; b) $2 R$; c) $4 R$; d) $\frac{5}{2} R$; e) $R$; f) $\frac{3}{2} R$.\\ În coordonate ($p, V$), ecuația transformării se scrie $p V^{-1}=b$, care este ecuația unei politrope cu indicele $n=-1$. Dar, $n=\frac{C-C_{p}}{C-C_{V}}=-1$, de unde rezultă $C=\frac{C_{p}+C_{V}}{2}=2 R$, sau, conform definiției căldurii molare:\\ $C=\frac{Q}{\nu \Delta T}=\frac{L+\Delta U}{\nu \Delta T}=\frac{\frac{\left(p_{1}+p_{2}\right)}{2}\left(V_{2}-V_{1}\right)+\nu C_{V} \Delta T}{\nu \Delta T}= \\ =\frac{p_{2} V_{2}-p_{1} V_{1}}{2 \nu \Delta T}+C_{V}=\frac{R}{2}+C_{V}=2 R$, deoarece, conform ecuației transformării, $p_{1} V_{2}=p_{2} V_{1}$. Răspuns corect b.\\

2007.A.8. Un obiect este aruncat vertical în sus. În momentul în care ajunge la jumătate din înălțimea maximă are o viteză de $10 \mathrm{~m} / \mathrm{s}$. Dacă $g=10 \mathrm{~m} / \mathrm{s}^{2}$, înălțimea maximă este: ( 8 pct.)\\ a) $10 \mathrm{~m}$; b) $100 \mathrm{~m}$; c) $15 \mathrm{~m}$; d) $25 \mathrm{~m}$; e) $5 \mathrm{~m}$; f) $20 \mathrm{~m}$.\\ Conform formulei lui Galilei, $h=\frac{v_{0}^{2}}{2 g}$ şi $\frac{h}{2}=\frac{v_{0}^{2}-v^{2}}{2 g}$, de unde $h=\frac{v^{2}}{g}=10 \mathrm{~m}$. Răspuns corect a.\\

2007.A.9. Prin conectarea unui rezistor având rezistenţa de $1,4 \mathrm{k} \Omega$ la o sursă de curent continuu, intensitatea curentului prin circuit devine de 29 ori mai mică decât intensitatea curentului de scurtcircuit. Rezistența internă a sursei este: (8 pct.)\\ a) $0,1 \Omega$; b) $5 \Omega$; c) $1 \Omega$; d) $10 \Omega$; e) $50 \Omega$; f) $2 \Omega$.\\ Conform condiției din enunț, $\frac{E}{R+r}=\frac{E}{29 r}$, de unde rezistența internă:\\ $r=\frac{R}{28}=50 \Omega$. Răspuns corect e.\\

2007.A.10. Două becuri electrice pe care scrie $40 \mathrm{~W}$, $220 \mathrm{~V}$ și $100 \mathrm{~W}$, $220 \mathrm{~V}$ sunt legate în serie şi alimentate la tensiunea de $220 \mathrm{~V}$. Puterea consumată în total de cele două becuri este: ( 6 pct.)\\ a) $\frac{200}{7} \mathrm{~W}$; b) $\frac{220}{7} \mathrm{~W}$; c) $\frac{100}{7} \mathrm{~W}$; d) $\frac{500}{7} \mathrm{~W}$; e) $\frac{400}{7} \mathrm{~W}$; f) $\frac{120}{7} \mathrm{~W}$.\\ $P_{\max}=\frac{U^{2}}{R_{1}+R_{2}}=\frac{U^{2}}{\frac{U^{2}}{P_{1}}+\frac{U^{2}}{P_{2}}}=\frac{P_{1} P_{2}}{P_{1}+P_{2}}=\frac{200}{7} \mathrm{~W}$. Răspuns corect a.\\

2007.A.11. Un corp cu masa de $2 \mathrm{~kg}$ este lansat în sus de-a lungul unui plan înclinat cu viteza inițială de $4 \mathrm{~m} / \mathrm{s}$. Corpul revine la baza planului înclinat cu o viteză egală cu jumătate din viteza inițială. Valoarea absolută a lucrului mecanic efectuat în timpul mişcării de forţa de frecare dintre corp şi plan este: ( 6 pct.)\\ a) $15 \mathrm{~J}$; b) $8 \mathrm{~J}$; c) $12 \mathrm{~J}$; d) $16 \mathrm{~J}$; e) $4 \mathrm{~J}$; f) $10 \mathrm{~J}$.\\ Conform teoremei de variație a energiei cinetice:\\ $L_{\text {frec}}=\Delta E_{c}=\frac{m}{2}\left(v^{2}-\frac{v^{2}}{4}\right)=\frac{3}{8} m v^{2}=12 \mathrm{~J}$. Räspuns corect c.\\

2007.A.12. Ecuația de mişcare a unui mobil este $x(t)=2+6 t-t^{2}$, în care mărimile fizice sunt exprimate în SI. După cât timp viteza mobilului este egală cu o treime din viteza sa inițială? ( 6 pct.)\\ a) $1,5 \mathrm{~s}$; b) $10 \mathrm{~s}$; c) $1 \mathrm{~s}$; d) $3 \mathrm{~s}$; e) $4 \mathrm{~s}$; f) $2 \mathrm{~s}$.\\ Ecuația vitezei este $v=x_{t}^{\prime}=6-2 t=2$, de unde $t=2 \mathrm{~s}$. Răspuns corect f.\\

2007.A.13. Un corp cu masa de $50 \mathrm{~kg}$ este tras pe o suprafaţă orizontală de către o forță $F$ care acționează sub unghiul $\alpha=60^{\circ}$ faţă de verticală. Dacă corpul se deplasează cu frecare, având viteza constantă şi sunt cunoscute valorile $g=10 \mathrm{~m} / \mathrm{s}^{2}$, $\mu=\frac{\sqrt{3}}{7}$, atunci valoarea forței $F$ este: ( 4 pct.)\\ a) $250 \mathrm{~N}$; b) $225 \mathrm{~N}$; c) $150 \mathrm{~N}$; d) $500 \mathrm{~N}$; e) $125 \mathrm{~N}$; f) $100 \mathrm{~N}$.\\ Din condiția de echilibru a forțelor, $F \sin \alpha=\mu(m g-F \cos \alpha)$, rezultă:\\ $F=\frac{\mu m g}{\sin \alpha+\mu \cos \alpha}=125 \mathrm{~N}$. Răspuns corect e.\\

2007.A.14. Legea inducției electromagnetice are următoarea expresie: ( 4 pct.)\\ a) $e=-\Delta \Phi \cdot \Delta t$; b) $e=\sqrt{\frac{\Delta \Phi}{\Delta t}}$; c) $e=-\frac{1}{2} \frac{\Delta \Phi}{\Delta t}$; d) $e=\frac{1}{2} \frac{\Delta \Phi}{\Delta t}$; e) $e=-\frac{\Delta \Phi}{\Delta t}$; f) $e=\frac{\Delta \Phi}{\Delta t}$.\\ Răspuns corect e.\\

2007.A.15. Impulsul unui corp este de $10 \mathrm{~N} \cdot \mathrm{s}$, iar energia sa cinetică de $10 \mathrm{~J}$. Masa corpului este: ( 4 pet.)\\ a) $6 \mathrm{~kg}$; b) $1 \mathrm{~kg}$; c) $14 \mathrm{~kg}$; d) $10 \mathrm{~kg}$; e) $15 \mathrm{~kg}$; f) $5 \mathrm{~kg}$.\\ Energia cinetică, $E_{c}=\frac{m v^{2}}{2}=\frac{p^{2}}{2 ~m}$, de unde $m=\frac{p^{2}}{2 E_{c}}=5 \mathrm{~kg}$. Răspuns corect f.\\

2007.A.16. Unitatea de măsură pentru randament este: ( 4 pct.)\\ a) $\frac{\mathrm{J} \cdot \mathrm{s}}{\mathrm{kg} \cdot \mathrm{m}}$; b) $\frac{\mathrm{J} \cdot \mathrm{s}^{2}}{\mathrm{~kg} \cdot \mathrm{m}^{2}}$; c) $\mathrm{W}$; d) $\mathrm{J} \cdot \mathrm{s}$; e) $\frac{\mathrm{N} \cdot \mathrm{m}}{\mathrm{J} \cdot \mathrm{s}}$; f) $\mathrm{J}$.\\ Randamentul este adimensional. Răspuns corect b.\\

2007.A.17. Dacă $L$ este lucrul mecanic efectuat de o sursă electrică pentru deplasarea sarcinii $q$ pe întregul circuit, atunci definiţia tensiunii electromotoare a sursei este: ( 4 pct.)\\ a) $E=\sqrt{\frac{L}{q}}$; b) $E=\frac{L}{q}$; c) $E=L \cdot q$; d) $E=\frac{L}{q^{2}}$; e) $E=\frac{L}{\sqrt{q}}$; f) $E=L \cdot q^{2}$.\\ Răspuns corect b.\\

2007.A.18. Un solenoid cu lungimea de $10 \mathrm{~cm}$ având 1000 spire şi permeabilitatea magnetică a miezului $\mu_{0}=4 \pi \cdot 10^{-7} \mathrm{~N} / \mathrm{A}^{2}$, este parcurs de un curent cu intensitatea de $1 \mathrm{~A}$. Inducția câmpului magnetic în interiorul său este: ( 4 pct.)\\ a) $0,001 \mathrm{~T}$; b) $4 \pi \cdot 10^{-5} \mathrm{~T}$; c) $4 \pi \cdot 10^{-4} \mathrm{~T}$; d) $4 \pi \cdot 10^{-3} \mathrm{~T}$; e) $0,02 \mathrm{~T}$; f) $4 \pi \cdot 10^{-7} \mathrm{~T}$.\\ $B=\mu_{0} \frac{N I}{l}=4 \pi \cdot 10^{-3} \mathrm{~T}$. Răspuns corect d.\\
