\section{Admitere F1 iulie 2009}

2009.A.1. Un mobil pleacă din repaus și se mişcă rectiliniu uniform accelerat. În secunda $n$ a mişcării mobilul parcurge o distanță de 3 ori mai mare decât în secunda anterioară. Valoarea lui $n$ este: (5 pct.)\\ a) $2$; b) $5$; c) $6$; d) $4$; e) $10$; f) $3$.\\ Spaţiul parcurs în a $n$-a secundă este egal cu diferenţa dintre spaţiul parcurs în $n$ secunde şi spaţiul parcurs în $n-1$ secunde, adică:\\ $s_{n}=\frac{1}{2} a n^{2}-\frac{1}{2} a(n-1)^{2}=\frac{1}{2} a(2 n-1)$;\\ Spaţiul parcurs în a $n-1$-a secundă este egal cu diferenţa dintre spaţiul parcurs în $n-1$ secunde şi spaţiul parcurs în $n-2$ secunde, adică:\\ $s_{n-1}=\frac{1}{2} a(n-1)^{2}-\frac{1}{2} a(n-2)^{2}=\frac{1}{2} a(2 n-3)$.\\ Din condiţia ca $s_{n}=3 s_{n-1}$, adică $2 n-1=3(2 n-3)$, rezultă $n=2$. Răspuns corect a.\\

2009.A.2. Un automobil, având viteza de $10 \mathrm{~m} / \mathrm{s}$ la baza unei pante de înclinare $3^{\circ}$ urcă panta fară motor. Știind coeficientul de frecare $\mu=0,05$ şi considerând $g=10 \mathrm{~m} / \mathrm{s}^{2}$, $\sin 3^{\circ} \approx 0,05$, $\cos 3^{\circ} \approx 1$, timpul după care viteza mobilului devine $5 \mathrm{~m} / \mathrm{s}$ este: ( 5 pct.)\\ a) $5 \mathrm{~s}$; b) $10 \mathrm{~s}$; c) $15 \mathrm{~s}$; d) $1 \mathrm{~min}$; e) $6 \mathrm{~s}$; f) $9 \mathrm{~s}$.\\ La urcarea pe planul înclinat fără motor, acceleraţia mobilului este negativă (mişcare uniform încetinită), adică:\\ $a=-g(\sin \alpha+\mu \cos \alpha)=-1 \frac{m}{s^{2}}$;\\ Din ecuaţia vitezei $v=v_{0}+a t$, rezultă:\\ $t=\frac{v-v_{0}}{a}=5 \mathrm{~s}$. Răspuns corect a.\\

2009.A.3. Viteza cu care trebuie aruncat pe verticală un corp de la înălțimea de $45 \mathrm{~m}$ pentru a ajunge pe sol cu o secundă mai târziu decât în cădere liberă este ($g=10 \mathrm{~m} / \mathrm{s}^{2}$): ( 5 pct.)\\ a) $1 \mathrm{~m} / \mathrm{s}$ în sus; b) $5 \mathrm{~m} / \mathrm{s}$ în jos; c) $2 \mathrm{~m} / \mathrm{s}$ în sus; d) $8,75 \mathrm{~m} / \mathrm{s}$ în jos; e) $3 \mathrm{~m} / \mathrm{s}$ în jos; f) $75 \mathrm{~m} / \mathrm{s}$ în sus.\\ Corpul trebuie aruncat în sus pentru ca acesta să ajungă pe sol mai târziu decât dacă ar fi lăsat să cadă liber. Condiţia problemei se scrie sub forma $t_{1}=t_{2}+1$, unde $t_{1}$ este intervalul de timp în care corpul aruncat în sus ajunge la sol, iar $t_{2}$ este intervalul de timp în care corpul lăsat liber ajunge la sol.\\ Intervalul de timp $t_{1}$ este compus din durata necesară corpului să ajungă la înălţimea maximă, adică $\frac{v_{0}}{g}$ şi durata căderii libere a corpului de la înălţimea maximă până la nivelul solului, adică $\sqrt{\frac{2}{g}\left(\frac{v_{0}^{2}}{2 g}+h\right)}$, $h$ fiind înălţimea de la care este aruncat corpul. Intervalul de timp $t_{2}=\sqrt{\frac{2 h}{g}}=3 \mathrm{~s}$.\\ Astfel, ecuaţia problemei devine:\\ $\frac{v_{0}}{g}+\sqrt{\frac{2}{g}\left(\frac{v_{0}^{2}}{2 g}+h\right)}=4$, din care rezultă că $v_{0}=8,75 \frac{m}{s}$, în sus. Răspuns corect f.\\

2009.A.4. Un automobil are în momentul începerii frânării viteza de $20 \mathrm{~m} / \mathrm{s}$. Considerând coeficientul de frecare dintre roți şi şosea $\mu=0,4$ şi $g=10 \mathrm{~m} / \mathrm{s}^{2}$, spațiul de frânare până la oprire este: ( 5 pct.)\\ a) $25 \mathrm{~m}$; b) $90 \mathrm{~m}$; c) $60 \mathrm{~m}$; d) $50 \mathrm{~m}$; e) $100 \mathrm{~m}$; f) $15 \mathrm{~m}$.\\ Acceleraţia de frânare este egală cu $a=\frac{F_{\text {frec}}}{m}=-\mu g$, iar din formula lui Galilei în care viteza finală este nulă, rezultă:\\ $d=\frac{v_{0}^{2}}{2 \mu}=50 \mathrm{~m}$.\\ Răspuns corect d.\\

2009.A.5. În cursul unui proces termodinamic în care presiunea este invers proporțională cu pătratul volumului, temperatura unui gaz ideal scade de 3 ori. În acest proces volumul gazului: ( 5 pct.)\\ a) scade de 2 ori; b) creşte de 9 ori; c) rămâne constant; d) creşte de 3 ori; e) scade de 3 ori; f) scade de 9 ori.\\ Ecuaţia transformării $p V^{2}=a$ se mai poate scrie sub forma $T V=b$, sau $T_{1} V_{1}=T_{2} V_{2}$, adică $\frac{V_{2}}{V_{1}}=\frac{T_{1}}{T_{2}}=3$. Răspuns corect d.\\

2009.A.6. Un gaz ideal monoatomic $\left(C_{V}=\frac{3}{2} R\right)$ se destinde după legea $p=a \cdot V$ cu $a=$ constant. Căldura molară în această transformare este: (5 pct.)\\ a) $6 R$; b) $2 R$; c) $0,5 R$; d) $R$; e) $3 R$; f) $5 R$.\\ Conform definiţiei, căldura molară este egală cu:\\ $C=\frac{Q}{\nu \Delta T}=\frac{L+\Delta U}{\nu \Delta T}=\frac{\operatorname{aria}+\nu C_{V} \Delta T}{\nu \Delta T}=\frac{\frac{1}{2}\left(p_{1}+p_{2}\right)\left(V_{1}+V_{2}\right)}{\nu\left(T_{2}-T_{1}\right)}+C_{V}= \\ =\frac{p_{2} V_{2}-p_{1} V_{1}}{2 \nu\left(\frac{p_{2} V_{2}}{\nu R}-\frac{p_{1} V_{1}}{\nu R}\right)}+\frac{3}{2} R=2 R$, unde am ţinut cont de ecuaţia transformării $p_{1} V_{1}=p_{2} V_{2}$ şi am calculat lucrul mecanic ca aria de sub dreapta cu ecuaţia $p=a V$ în coordonate $(p, V)$. Răspuns corect b.\\

2009.A.7. Unitatea de măsură a forţei în S. I. este: (5 pct.)\\ a) $\mathrm{kg} \cdot \mathrm{m}^{2} \cdot \mathrm{s}^{-2}$; b) $\mathrm{N} \cdot \mathrm{m}$; c) $\mathrm{N}$; d) $\mathrm{N} / \mathrm{m}^{2}$; e) $\mathrm{kg} \cdot \mathrm{m} \cdot \mathrm{s}^{-1}$; f) $\mathrm{m} \cdot \mathrm{s}^{-2}$.\\ Răspuns corect c.\\

2009.A.8. Unitatea de măsură în S. I. a căldurii specifice este: (5 pct.)\\ a) $\mathrm{J} / \mathrm{kg}$; b) $\mathrm{J} /(\mathrm{mol} \cdot \mathrm{K})$; c) $\mathrm{J}$; d) $\mathrm{J} / \mathrm{K}$; e) $\mathrm{J} \cdot \mathrm{K}$; f) $\mathrm{J}(\mathrm{kg} \cdot \mathrm{K})$.\\ Răspuns corect f.\\

2009.A.9. Utilizând notațiile din manualele de fizică, expresia legii lui Hooke este: (5 pct.)\\ a) $\sigma=\frac{\varepsilon}{E}$; b) $\frac{\Delta l}{l_{0}}=\frac{1}{E} \frac{F}{S_{0}}$; c) $F=-k x^{2}$; d) $\Delta l \cdot l_{0}=E \frac{F}{S_{0}}$; e) $\Delta l=E I_{0} \frac{S_{0}}{F}$; f) $F=m \cdot a$. Răspuns corect b.\\

2009.A.10. Într-un circuit simplu, tensiunea la bornele bateriei este de $3 \mathrm{~V}$. Mărind rezistența exterioară de 3 ori, tensiunea la borne creşte cu $20 \%$. În aceste condiții t. e. m. a bateriei este: ( 5 pct.)\\ a) $12 \mathrm{~V}$; b) $10 \mathrm{~V}$; c) $4 \mathrm{~V}$; d) $20 \mathrm{~V}$; e) $15 \mathrm{~V}$; f) $9 \mathrm{~V}$.\\ Din expresia tensiunii la borne $U=I R=\frac{E R}{R+r}$ şi $1,2 U=\frac{3 E R}{3 R+r}$, rezultă $r=\frac{R}{3}$ şi $E=\frac{U(R+r)}{R}=4 \mathrm{~V}$. Răspuns corect c.\\

2009.A.11. Utilizând notațiile din manualele de fizică, expresia principiului întâi al termodinamicii este: (5 pct.)\\ a) $C_{p}-C_{V}=R$; b) $\eta=\frac{T_{1}-T_{2}}{T_{1}}$; c) $\Delta U=Q / L$; d) $\eta=\frac{Q_{1}-\left|Q_{2}\right|}{Q_{1}}$; e) $\Delta U=Q-L$; f) $\Delta Q=U+L$.\\ Răspuns corect e.\\

2009.A.12. Două conductoare cu aceeaşi secțiune transversală având rezistivitățile la o temperatură de referință $\rho_{01}=6 \cdot 10^{-5} \quad \Omega \cdot \mathrm{m}$ şi respectiv $\rho_{02}=3 \cdot 10^{-6} \quad \Omega \cdot \mathrm{m}$ şi coeficienții termici ai rezistivității $\alpha_{1}=-5 \cdot 10^{-4} \operatorname{grad}^{-1}$ şi respectiv $\alpha_{2}=4 \cdot 10^{-4} \operatorname{grad}^{-1}$ se leagă în serie. Se neglijează efectele de dilatare termică. Lungimea primului conductor este $l_{1}=1 \mathrm{~m}$. Pentru ca rezistența grupării să nu varieze cu temperatura, lungimea $l_{2}$ a celui de-al doilea conductor este: ( 5 pct.)\\ a) $5 \mathrm{~m}$; b) $100 \mathrm{~m}$; c) $25 \mathrm{~m}$; d) $2 \mathrm{~m}$; e) $80 \mathrm{~m}$; f) $50 \mathrm{~m}$.\\ Conform condiţiei problemei,\\ $R_{01}+R_{02}=R_{t 1}+R_{t 2}$ sau $R_{01}+R_{02}=R_{01}\left(1+\alpha_{1} t\right)+R_{02}\left(1+\alpha_{2} t\right)$, de unde $R_{01} \alpha_{1}=-R_{02} \alpha_{2}$ sau $\rho_{01} \frac{l_{1}}{S} \alpha_{1}=-\rho_{02} \frac{l_{12}}{S} \alpha_{2}$, adică $l_{2}=-\frac{\rho_{01} l_{1} \alpha_{1}}{\rho_{02} \alpha_{2}}=25 \mathrm{~m}$. Răspuns corect c.\\

2009.A.13. Un generator produce aceeaşi putere electrică într-un rezistor cu rezistența de $9 \Omega$ sau într-un rezistor cu rezistența de $16 \Omega$. Rezistența internă a generatorului este: ( 5 pct.)\\ a) $4 \Omega$; b) $12 \Omega$; c) $6 \Omega$; d) $10 \Omega$; e) $24 \Omega$; f) $2 \Omega$.\\ Conform condiţiei problemei,\\ $I_{1}^{2} R_{1}=I_{2}^{2} R_{2}$ sau $\frac{E^{2} R_{1}}{\left(R_{1}+r\right)^{2}}=\frac{E^{2} R_{1}}{\left(R_{2}+r\right)^{2}}$, adică $R_{1}\left(R_{2}+r\right)^{2}=R_{2}\left(R_{1}+r\right)^{2}$, de unde $r=\sqrt{R_{1} R_{2}}=12 \Omega$. Răspuns corect b.\\

2009.A.14. Unitatea de măsură a rezistivității electrice în S. I. este: (5 pct.)\\ a) $\mathrm{V}$; b) $\Omega \cdot \mathrm{m}$; c) $\Omega$; d) $\Omega / \mathrm{m}$; e) $\mathrm{A}$; f) $\Omega \cdot m^{2}$.\\ Răspuns corect b.\\

2009.A.15. Printr-un conductor de lungime $100 \mathrm{~m}$ şi sectiune $1 \mathrm{~mm}^{2}$ trece un curent de $1,6 \mathrm{~A}$ dacă la capetele lui se aplică o tensiune de $4 \mathrm{~V}$. Rezistivitatea materialului din care este confecționat conductorul este: (5 pct.)\\ a) $2 \cdot 10^{-8}$; b) $2,5 \cdot 10^{-6} \Omega \cdot \mathrm{m}$; c) $4 \cdot 10^{-8} \Omega \cdot \mathrm{m}$; d) $2,5 \cdot 10^{-8} \Omega \cdot \mathrm{m}$; e) $3 \cdot 10^{-8} \Omega$; f) $5 \cdot 10^{-8} \Omega / \mathrm{m}$.\\ Din legea lui Ohm, $U=I R=I \frac{\rho l}{S}$ rezultă $\rho=\frac{U S}{I l}=2,5 \cdot 10^{-8} \Omega \cdot \mathrm{m}$. Răspuns corect d.\\

2009.A.16. O maşină termică ideală funcționează după un ciclu Carnot între temperaturile $T_{1}=400 \mathrm{~K}$ şi $T_{2}=300 \mathrm{~K}$. Știind că în timpul unui ciclu maşina primeşte căldura $Q_{1}=400 \mathrm{~kJ}$, lucrul mecanic efectuat de maşină în timpul unui ciclu este: (5 pct.)\\ a) $100 \mathrm{~J}$; b) $20000 \mathrm{~J}$; c) $400 \mathrm{~J}$; d) $125 \mathrm{~kJ}$; e) $100 \mathrm{~kJ}$; f) $420 \mathrm{~kJ}$.\\ Randamentul unei maşini termice ideale este:\\ $\eta=\frac{L}{Q}=1-\frac{T_{2}}{T_{1}}$;\\ De aici lucrul mecanic este:\\ $L=Q_{1}(1-\frac{T_{2}}{T_{1}})=100 \mathrm{~kJ}$. Răspuns corect e.\\

2009.A.17. Pentru a încălzi izobar cu $5 \mathrm{~K}$ o cantitate de $10 \mathrm{~moli}$ de hidrogen se transmite gazului căldura $Q=915 \mathrm{~J}$. Știind că $R=8,3 \mathrm{~J} /(\mathrm{mol} \cdot \mathrm{K})$, variația energiei interne a gazului în procesul considerat este: (5 pct.)\\ a) $508 \mathrm{~J}$; b) $412 \mathrm{~J}$; c) $500 \mathrm{~J}$; d) $550 \mathrm{~J}$; e) $485 \mathrm{~J}$; f) $512 \mathrm{~J}$.\\ Conform primului principiu al termodinamicii:\\ $\Delta U=Q-L=Q-\nu R \Delta T=500 \mathrm{~J}$. Răspuns corect c.\\

2009.A.18. Utilizând notaţiile din manualele de fizică, expresia legii lui Ohm pentru circuitul simplu este: (5 pct.)\\ a) $E=\frac{I}{R+r}$; b) $I=\frac{E}{R}$; c) $P=U \cdot I$; d) $I=\frac{E}{R+r}$; e) $U=R \cdot I$; f) $I=\frac{E}{r}$.\\ Răspuns corect d.\\