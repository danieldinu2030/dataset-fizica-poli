% iulie 2003, Admitere UPB, Fizică FA. Enunțuri şi rezolvare %

\begin{enumerate}
  \item Pe un plan orizontal un corp de masă $m_{1}$ ciocneşte elastic un corp de masă $m_{2}$ aflat în repaus. În urma ciocnirii, cele două corpuri se deplasează cu aceeaşi viteză, în sensuri opuse. Raportul $\frac{m_{2}}{m_{1}}$ este: ( 8 pct.)\\
a) 3 ; b) 4 ; c) $\frac{1}{3}$; d) 2 ; e) 7 ; f) 1 .
  \item Randamentul unei maşini termice, funcționând după un ciclu Carnot cu gaz ideal este $\eta=64 \%$. Raportul (subunitar) al vitezelor termice ale moleculelor de gaz corespunzătoare temperaturilor extreme ale ciclului este: (8 pct.)\\
a) 0,8 ; b) 0,89 ; c) 0,64 ; d) 0,6 ; e) 0,4 ; f) 0,5 .
  \item Două baterii A şi B cu t. e. m. $E_{A}=6 \mathrm{~V}$ şi $E_{B}=3 \mathrm{~V}$, având rezistențele interne $r_{A}=1 \Omega$, respectiv $r_{B}=2 \Omega$ sunt legate în serie la bornele unui rezistor de rezistență $R$. Pentru ce valoare a rezistenței $R$, tensiunea la bornele bateriei B va fi nulă? (8 pct.)\\
a) $6 \Omega$; b) $2 \mathrm{k} \Omega$; c) $3 \Omega$; d) $1,5 \Omega$; e) $2 \Omega$; f) $3 \mathrm{k} \Omega$.
  \item Densitatea unui gaz ideal aflat la temperatura $\mathrm{T}_{1}=300 \mathrm{~K}$ este $\rho_{1}=1 \mathrm{~kg} / \mathrm{m}^{3}$. Care va fi densitatea gazului la temperatura $\mathrm{T}_{2}=400 \mathrm{~K}$, presiunea rămânând constantă? ( 6 pct.)\\
a) $1,75 \mathrm{~kg} / \mathrm{m}^{3}$; b)\\
b) $0,65 \mathrm{~kg} / \mathrm{m}^{3}$; c) $0,75 \mathrm{~kg} / \mathrm{m}^{3}$; d\\
d) $0,5 \mathrm{~kg} / \mathrm{m}^{3}$; e) $1,75 \mathrm{~kg} / \mathrm{m}^{3}$; f) $0,86 \mathrm{~g} / \mathrm{cm}^{3}$.
  \item Sub acțiunea unei forțe $\mathrm{F}=25 \mathrm{~N}$, un resort elastic se comprimă $\mathrm{cu} \mathrm{x}=4 \mathrm{~cm}$. Ce energie potențială dobândeşte resortul în urma acestei comprimări? ( $\mathbf{6}$ pct.)\\
a) $0,5 \mathrm{~J}$; b) $8 \mathrm{~N} \cdot \mathrm{~m}$; c) 5 J ; d) $12,5 \mathrm{~J}$; e) 1 J ; f) $7,4 \mathrm{~N}$.
  \item O baterie cu t. e. m. $E=24 \mathrm{~V}$ are curentul de scurtcircuit $I_{s}=60 \mathrm{~A}$. Ce rezistență are un consumator care conectat la această baterie face ca tensiunea la borne să fie $\mathrm{U}=22 \mathrm{~V}$ ? ( 6 pct.)\\
а) $4,2 \Omega$;\\
b) $4,4 \Omega$; c) $8,8 \Omega$;\\
d) $2,2 \Omega$; e) $6,5 \Omega$; f) $3,4 \Omega$.
  \item Doi moli de gaz cântăresc 64 g . Masa molară a gazului este: ( $\mathbf{4}$ pct.)\\
a) $128 \frac{\mathrm{~kg}}{\mathrm{kmol}}$; b) $54 \frac{\mathrm{~kg}}{\mathrm{kmol}}$; c) $12 \frac{\mathrm{~g}}{\mathrm{~mol}}$; d) $3,2 \frac{\mathrm{~kg}}{\mathrm{kmol}}$; e) $32 \frac{\mathrm{~kg}}{\mathrm{kmol}}$; f) $38 \frac{\mathrm{~kg}}{\mathrm{kmol}}$.
  \item Într-o mişcare uniform încetinită, viteza unui mobil la un anumit moment este de $40 \mathrm{~m} / \mathrm{s}$. Dacă după 8 s mobilul se opreşte, accelerația de frânare are mărimea: ( $\mathbf{4}$ pct.)\\
a) $3,2 \mathrm{~m} / \mathrm{s}^{2}$; b) $6 \mathrm{~m} / \mathrm{s}^{2}$; c) $4,8 \mathrm{~m} / \mathrm{s}^{2}$; d) $0,2 \mathrm{~m} / \mathrm{s}^{2}$; e) $3 \mathrm{~m} / \mathrm{s}^{2}$; f) $5 \mathrm{~m} / \mathrm{s}^{2}$.
  \item Inducţia magnetică pe axul unei bobine foarte lungi, parcursă de curent continuu este: ( $\mathbf{4}$ pct.)\\
a) $\frac{\mathrm{NI}}{\mu \mathrm{l}}$; b) $\frac{\mu \mathrm{NI}}{2 \mathrm{R}}$; c) $\frac{\mu \mathrm{I}}{\mathrm{Nl}}$; d) $\frac{\mu \mathrm{I}}{2 \mathrm{R}}$; e) $\frac{\mu \mathrm{NI}}{1}$; f) $\frac{\mathrm{N}^{2} \mathrm{I}}{\mu \mathrm{l}}$.
  \item Două bile $A$ şi $B$ de mase $m_{A}=100 \mathrm{~g}$ şi $m_{B}=200 \mathrm{~g}$ se ciocnesc plastic. În urma ciocnirii bilele se opresc. Dacă bila $A$ avea viteza $v_{A}=5 \mathrm{~m} / \mathrm{s}$, bila $B$ avea viteza: ( $\mathbf{4}$ pct.)\\
a) $4,5 \mathrm{~m} / \mathrm{s}$; b) $7,5 \mathrm{~m} / \mathrm{s}$; c) $2,5 \mathrm{~m} / \mathrm{s}$; d) $10 \mathrm{~m} / \mathrm{s}$; e) $10,5 \mathrm{~m} / \mathrm{s}$; f) $8 \mathrm{~m} / \mathrm{s}$.
  \item În SI puterea se măsoară în (4 pct.)\\
a) $J$; b) $J / s^{2}$; c) $N$; d) $W$; e) $J \cdot s$; f) $N \cdot m$.
  \item În SI constanta elastică a unui resort are ca unitate de măsură ( $\mathbf{4}$ pct.)\\
a) $\mathrm{J} / \mathrm{m}$; b) $\mathrm{N} \cdot \mathrm{m}$; c) $\mathrm{N} / \mathrm{m}$; d) $\mathrm{kg} \cdot \mathrm{m}$; e) $\mathrm{N} / \mathrm{m}^{2}$; f) $\mathrm{N} \cdot \mathrm{m}^{2}$.
  \item Fie un circuit de curent continuu alcătuit dintr-o sursă cu t. e. m. $E=102 \mathrm{~V}$ şi un rezistor cu rezistența $\mathrm{R}=1 \mathrm{k} \Omega$. Dacă tensiunea la borne este $\mathrm{U}=100 \mathrm{~V}$, rezistența internă a sursei are valoarea: ( 4 pct.)\\
a) $2 \Omega$; b) $60 \Omega$; c) $10 \Omega$; d) $20 \Omega$; e) $10 \mathrm{k} \Omega$; f) $20 \mathrm{k} \Omega$.
  \item Un corp care primeşte căldura $Q=8 \mathrm{~kJ}$ îşi măreşte temperatura $\mathrm{cu} \Delta T=40 \mathrm{~K}$. Capacitatea calorică a corpului este: (4 pct.)\\
a) $420 \mathrm{~J} / \mathrm{K}$; b) $320 \mathrm{~J} / \mathrm{K}$; c) $200 \mathrm{~J} / \mathrm{K}$; d) $3 \cdot 10^{3} \mathrm{~J} / \mathrm{K}$; e) $50 \mathrm{~J} / \mathrm{K}$; f) $80 \mathrm{~J} / \mathrm{K}$.
  \item Expresia forței electromagnetice pentru un conductor filiform rectiliniu parcurs de curent şi aflat în câmp magnetic uniform este: (4 pct.)\\
a) $I \overrightarrow{1} \cdot \vec{B}$; b) $I(\overrightarrow{1} \times \vec{B})$; c) $\vec{I}(\overrightarrow{1} \times \vec{B})$; d) $\operatorname{IB}^{2} \overrightarrow{1}$; e) $I(\vec{B} \times \overrightarrow{1})$; f) $\mathrm{BI}^{2} 1$.
  \item Se leagă în paralel doi rezistori având rezistențele $R_{1}=6 \mathrm{k} \Omega$ şi $R_{2}=4 \mathrm{k} \Omega$. Rezistența echivalentă este: (4 pct.)\\
a) $3,5 \mathrm{k} \Omega$\\
b) $24 \mathrm{k} \Omega$; c)\\
$6,2 \mathrm{k} \Omega$\\
d) $10 \mathrm{k} \Omega$; e\\
e) $2,4 \mathrm{k} \Omega$; f) $48 \mathrm{k} \Omega$.
  \item Randamentul unei maşini termice este: (4 pct.)\\
a) $\frac{Q_{1}}{L}$;\\
b) $\frac{L-Q_{1}}{Q_{1}}$; c) $\frac{Q_{2}}{Q_{1}}$; d)\\
) $Q_{1}-L$; e) $\frac{L}{Q_{1}}$; f) $\frac{Q_{1}-L}{L}$
  \item Care dintre relațiile de mai jos reprezintă ecuația transformării adiabatice a unui gaz ideal? (4 pct.)\\
a) $\frac{V}{T}=$ const ;\\
b) $p V=$ const ; c) $\frac{p}{T}=$ const ;\\
d) $\frac{\mathrm{p}_{1} \mathrm{~V}_{1}}{\mathrm{~T}_{1}}=\frac{\mathrm{p}_{2} \mathrm{~V}_{2}}{\mathrm{~T}_{2}} ;$ e) $\mathrm{TV}^{\gamma-1}=$ const $;$ f\\
f) $p V=v R T$.
\end{enumerate}

\section*{Rezolvare subiecte admitere Politehnică 2003}
\begin{enumerate}
  \item Legile de conservare ale impulsului și energiei cinetice în ciocnirea elastică din enunț se scriu:
\end{enumerate}

$$
m_{1} v+0=-m_{1} v_{1}+m_{2} v_{1}
$$

şi $\quad \frac{1}{2} m_{1} v^{2}+0=\frac{1}{2} m_{1} v_{1}^{2}+\frac{1}{2} m_{2} v_{1}^{2}$, relații care se mai pot scrie sub forma\\
$m_{1}\left(v+v_{1}\right)=m_{2} v_{1}$\\
şi $\quad m_{1}\left(v^{2}-v_{1}^{2}\right)=m_{2} v_{1}^{2}$.\\
După împărțirea ultimelor relații, membru cu membru, obținem că $v=2 v_{1}$. Înlocuind acest rezultat în legea conservării impulsului rezultă raportul $\frac{m_{2}}{m_{1}}=3$.

\section*{Răspuns corect $\boldsymbol{a}$}
\begin{enumerate}
  \setcounter{enumi}{1}
  \item Din expresia randamentului ciclului Carnot, $\eta=1-\frac{T_{2}}{T_{1}}$ şi expresiile vitezelor termice, $v_{T_{1}}=\sqrt{\frac{3 R T_{1}}{\mu}}$ şi respectiv $v_{T_{2}}=\sqrt{\frac{3 R T_{2}}{\mu}}$, rezultă raportul $\frac{v_{T_{2}}}{v_{T_{1}}}=\sqrt{\frac{T_{2}}{T_{1}}}=\sqrt{1-\eta}=0,6$. Răspuns corect $\boldsymbol{d}$
  \item Conform teoremei a doua a lui Kirchhoff, intensitatea curentului prin circuit este egală cu $I=\frac{E_{A}+E_{B}}{r_{A}+r_{B}+R}$, iar din condiția ca tensiunea la bornele bateriei B să fie nulă, adică $U_{B}=E_{B}-I r_{B}=\frac{E_{A}\left(r_{A}+R\right)-E_{B} r_{B}}{r_{A}+r_{B}+R}=0$, rezultă $R=\frac{E_{A} r_{B}-E_{B} r_{A}}{E_{B}}=3 \Omega$. Răspuns corect c
  \item Densitatea gazului ideal la temperatura $T_{1}$ este $\rho_{1}=\frac{p \mu}{R T_{1}}$, iar la temperatura $T_{2}$ este $\rho_{2}=\frac{p \mu}{R T_{2}}$, astfel că $\mathrm{kg} / \mathrm{m}^{3}$. Răspuns corect $c$
  \item Cu ajutorul expresiei forței elastice, $F=-k x$, putem scrie expresia energiei potențiale, $E_{p}=\frac{1}{2} k x^{2}=\frac{1}{2}|F| x=0,5 \mathrm{~J}$. Răspuns corect $\boldsymbol{a}$
  \item din expresia intensității curentului de scurtcircuit, $I_{s c}=\frac{E}{r}$, de unde $r=\frac{E}{I_{s c}}$ şi din expresia tensiunii la bornele circuitului, $U=E-I r=\frac{E R}{R+r}$, rezulră $R=\frac{U E}{(E-U) I_{s c}}=4,4 \Omega$. Răspuns corect $\boldsymbol{b}$
  \item Masa molară a gazului este $\mu=\frac{m}{v}=32 \cdot 10^{-3} \mathrm{~kg} / \mathrm{mol}$. Răspuns corect $\boldsymbol{e}$
  \item Din ecuația vitezei, $v=v_{0}-a t=0$, rezultă accelerația $a=\frac{v_{0}}{t}=5 \mathrm{~m} / \mathrm{s}^{2}$. Răspuns corect $\boldsymbol{f}$
  \item Răspuns corect $\boldsymbol{e}$
  \item Din legea de conservare ale impulsului în ciocnirea plastică $m_{A} v_{A}-m_{B} v_{B}=0$, rezultă viteza $v_{B}=\frac{m_{A} v_{A}}{m_{B}}=2,5 \mathrm{~m} / \mathrm{s}$. Răspuns corect $\boldsymbol{c}$
  \item Răspuns corect $\boldsymbol{d}$
  \item Răspuns corect c
  \item Din expresia tensiunii la bornele circuitului, $U=E-I r=\frac{E R}{R+r}$, rezultă rezistența internă a sursei $r=\frac{(E-U) R}{U}=20 \Omega$. Răspuns corect $\boldsymbol{d}$
  \item Conform definiției, capacitatea calorică a corpului este $C=\frac{Q}{\Delta T}=200 \mathrm{~J} / \mathrm{K}$. Răspuns corect $\boldsymbol{c}$
  \item Răspuns corect $\boldsymbol{b}$
  \item La legarea rezistoarelor în paralel, rezistența echivalentă este egală cu $R_{\text {ech }}=\frac{R_{1} R_{2}}{R_{1}+R_{2}}=2,4 \mathrm{k} \Omega$. Răspuns corect e
  \item Răspuns corect $\boldsymbol{e}$
  \item Răspuns corect $\boldsymbol{e}$.
\end{enumerate}

\end{document}